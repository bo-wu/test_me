%%%%%%%%%%%%%%%%%%%%%%%%%%%%%%%%%%%%%%%%%
% Wilson Resume/CV
% XeLaTeX Template
% Version 1.0 (22/1/2015)
%
% This template has been downloaded from:
% http://www.LaTeXTemplates.com
%
% Original author:
% Howard Wilson (https://github.com/watsonbox/cv_template_2004) with
% extensive modifications by Vel (vel@latextemplates.com)
%
% License:
% CC BY-NC-SA 3.0 (http://creativecommons.org/licenses/by-nc-sa/3.0/)
%
%%%%%%%%%%%%%%%%%%%%%%%%%%%%%%%%%%%%%%%%%

%----------------------------------------------------------------------------------------
%	PACKAGES AND OTHER DOCUMENT CONFIGURATIONS
%----------------------------------------------------------------------------------------

\documentclass[20pt]{article} % Default font size

\input{structure.tex} % Include the file specifying document layout

%----------------------------------------------------------------------------------------

\begin{document}

%----------------------------------------------------------------------------------------
%	NAME AND CONTACT INFORMATION
%----------------------------------------------------------------------------------------

\title{吴博} % Print the main header
%------------------------------------------------
%\makeatletter\def\@captype{table}\makeatother
\begin{minipage}{0.5\textwidth}
\centering
%\renewcommand{\linespread}{1.5}
\begin{tabbing}% Enables tabbing
\hspace{3cm} \= \hspace{4cm} \= \kill % Spacing within the block
{\bf 地址} \> 国防科学技术大学博士生队,\\% Address line 1
\> 湖南省 长沙市, 410073 \\% Address line 2
{\bf 出生年月} \> 1987年10月 \\ % Date of birth
{\bf 籍贯} \> 河南周口 \\% Nationality
{\bf 联系方式} \> 13808453365 \\ % Date of birth
{\bf 电子邮箱} \> bo_wu@nudt.edu.cn \\ \\% Nationality
{\bf 研究方向} \> 虚拟现实、图形图像处理 \\ % Date of birth
%{\bf 毕业日期} \> 2015年6月 \\ % Date of birth
\end{tabbing}%}
\end{minipage}
%\makeatletter\def\@captype{figure}\makeatother
\begin{minipage}{0.5\textwidth}
\hspace{3cm}
%\begin{figure}
 % \section{}
  \includegraphics[width=0.4\textwidth]{me}\\
%\end{figure}
\end{minipage}
%\begin{tabbing} % Enables tabbing
%\hspace{3cm} \= \hspace{4cm} \= \kill % Spacing within the block
%{\bf Home Phone} \> +0 (000) 111 1111 \\ % Home phone
%{\bf Mobile Phone} \> +0 (000) 111 1112 \\ % Mobile phone
%{\bf Email} \> \href{mailto:john@smith.com}{john@smith.com} \\ % Email address
%\end{tabbing}}
%----------------------------------------------------------------------------------------
%	PERSONAL PROFILE
%----------------------------------------------------------------------------------------
%\section{个人简介}
\begin{flushleft}
我是国防科学技术大学计算机学院的博士研究生,指导老师为熊岳山教授。我的主要
研究方向为三维几何建模、分析,和二维图像分析,其他研究兴趣包括分形几何建模
和变形(硕士课题)、算法设计和机器学习等。

读博期间,我曾在美国圣路易斯华盛
顿大学联合培养两年,师从Tao Ju教授。迄今,以第一作者身份发表SCI两篇,一篇为
计算机图形学A类期刊(ACM Transaction on Graphics),一篇为B类期刊(Computer
Graphics Forum),另有两篇第一作者的EI论文。以第一作者身份开发软件两个,并参
与若干个项目开发。
\end{flushleft}
%----------------------------------------------------------------------------------------
%	EDUCATION SECTION
%----------------------------------------------------------------------------------------

\section{教育经历}
\tabbedblock{
\bf{2012.03- } \> {国防科学技术大学} \> 计算机科学与技术 \>  博士生}
\tabbedblock{
\bf{2012.09-2013.09} \> {奥地利科学技术研究院} \> 计算机科学与技术 \>  联合培养}
\tabbedblock{
\bf{2009.09-2011.12} \> {国防科学技术大学} \> 计算机科学与技术 \>  硕士}
\tabbedblock{
\bf{2005.09-2009.06} \> {国防科学技术大学} \> 计算机科学与技术 \>  学士%\>First Class - 80\% Average\\
%\>\+
%\textit{Third Year Project - 89\% awarded `Project of the Year 2007'}
}
%------------------------------------------------


%----------------------------------------------------------------------------------------
%	EMPLOYMENT HISTORY SECTION
%----------------------------------------------------------------------------------------

\section{科研经历}
\job
{2014 - }{}
{虚拟手术中的触觉形变计算与渲染关键技术研究}
{国家自然科学基金项目,61379103}
{参与者}
%------------------------------------------------

\job
{2012 - }{2014}
{Reconstructing Geometrically and Topologically Correct Models}
{National Science Foundation, 0846072, U.S}
{参与者}

%------------------------------------------------
%\job
%{Oct 2009 -}{Sep 2010}
%{Initech Inc., Otis St, CA 94025, United States}
%{http://www.initech.com}
%{Analyst}
%{Lorem ipsum dolor sit amet, consectetur adipiscing elit. Duis elementum nec dolor:
%\begin{itemize-noindent}
%\item{Developed spreadsheets for risk analysis on exotic derivatives on a wide array of commodities (ags, oils, precious and base metals.)}
%\item{Managed blotter and secondary trades on structured notes, liaised with Middle Office, Sales and Structuring for bookkeeping.}
%\end{itemize-noindent}
%Lorem ipsum dolor sit amet, consectetur adipiscing elit. Duis elementum nec dolor sed sagittis.}
%----------------------------------------------------------------------------------------
%	IT/COMPUTING SKILLS SECTION
%----------------------------------------------------------------------------------------
\section{发表论文}

[1]   Yixin Zhuang, Ming Zou, Nathan Carr, Tao Ju. {A general and efficient method for finding cycles in 3D curve networks}. ACM Transaction on Graphics (SIGGRAPH Asia 2013). (SCI检索,CCF计算机图形学A类期刊,影响因子3.725)\\%\href{http://dl.acm.org/citation.cfm?doid=2508363.2508423}


[2] Yixin Zhuang, Ming Zou, Nathan Carr, Tao Ju. {Anisotropic geodesics for live-wire mesh segmentation}. Computer Graphics Forum (Pacific Graphics 2014).(SCI检索,\\CCF计算机图形学B类期刊,影响因子1.595)\\%\href{http://onlinelibrary.wiley.com/doi/10.1111/cgf.12479/full}


[3]   庄一新,熊岳山. 一种新的保持分形特征的分形变形方法. 国防科大学报, 2012. (EI检索)\\


[4]   Zhuang Yixin, Xiong Yueshan, Liu Fayao. IFS fractal morphing based on coarse convex-hull. ITAIC, 2011. (EI检索)

%[1]	Yixin Zhuang, Ming Zou, Nathan Carr, Tao Ju. Anisotropic geodesics for live-wire mesh segmentation. Computer Graphics Forum, 2014.
%
%[2]	Yixin Zhuang, Ming Zou, Nathan Carr, Tao Ju. A general and efficient method for finding cycles in 3D curve networks. ACM Transaction on Graphics, 2013.
%
%[3]	庄一新,熊岳山. 一种新的保持特征的分形变形方法. 国防科大学报, 2013.
%
%[4]	Zhuang Yixin, Xiong Yueshan, Liu Fayao. IFS fractal morphing based on coarse convex-hull. ITAIC, 2011.

\section{科研能力}

\begin{minipage}{0.5\textwidth}
\centering
\skillgroup{程序设计语言}
{
\textit{C, C++} - 高效算法设计\\
\textit{Python, Json} - 脚本语言\\
\textit{Matlab, Mathematica} - 科学计算、原型开发\\
\textit{Git} - 代码管理
}
\end{minipage}
\begin{minipage}{0.5\textwidth}
\skillgroup{语言}
{
\textit{英语} -   流利,在美国学习生活两年
}
\skillgroup{开发工具}
{
\textit{wxWidget, QT} - 界面设计\\
\textit{QT creator,Visual Studio} - 开发环境
}
\end{minipage}

\section{软件开发}

\begin{minipage}{0.5\textwidth}
\skillgroup{1. Livewire3D}
{
\center
  \includegraphics[width= \textwidth]{livewire}
}
\end{minipage}
\begin{minipage}{0.5\textwidth}
交互式的三维网格分割工具。允许用户在
三维模型上画线,算法会自动对线条进行
一系列优化,比如使线条更加吻合模型的
特征、更加光滑等。然后利用网格上线条
将完整三维模型分割成一片片子网格面片
。这一工具允许用户显式的指定分割的面
片区域,对于进行纹理贴图,模型重网格
化等,有非常重要的意义。见[2]\\%,链接
%\href{http://www.cse.wustl.edu/~zoum/projects/Livewire/}{\small http://www.cse.wustl.edu/~zoum/projects/Livewire/}
作者为唯一开发人。
\end{minipage}


\begin{minipage}{0.5\textwidth}
\skillgroup{2. CycleDisc}
{
\center
  \includegraphics[width= \textwidth]{cycledisc}
}
\end{minipage}
\begin{minipage}{0.5\textwidth}
三维几何建模工具,具体地说,既将三维
曲线网格转化成三维曲面,曲面是多个面
片组成,每个面片由若干条曲线包围。用
户可以交互式地修改自动算法生成的结果
,包括,选择并修改曲线的属性,或者通
过选择多条曲线来指定某个具体面片的位
置,然后算法会实时更新结果。见[1]
作者为唯一开发人。\\
\end{minipage}


\begin{minipage}{0.5\textwidth}
\skillgroup{3. GeoTopo}
{
\center
  \includegraphics[width= 0.8\textwidth]{GeoTopo}
}
\end{minipage}
\begin{minipage}{0.5\textwidth}
三维模型的自动匹配。输入两个三维几何
模型,模型已经分割为各个子部分,算法
自动寻找两个模型间子部分的对应。此项
目为{Simon Fraser University StarLab}
(三维网格的建模、处理和分析软件)的子项
目。\\
作者为第二开发人,负责搜索算法设计。
\end{minipage}
\\
\\


\begin{minipage}{0.5\textwidth}
\skillgroup{4. 肝脏虚拟手术软件}
{
\center
  \includegraphics[width= 0.8\textwidth]{surgery}
}
\end{minipage}
\begin{minipage}{0.5\textwidth}
肝脏病变组织摘除手术。输入肝的几何模
型,交互式地标定病变位置的边界,然后
利用“手术刀”沿着病变处切割,直到分离
出整个病变组织。该软件包含多种技术,
如几何模型的切割,变形,渲染等。
作者参与了项目研究及部分实现。
\end{minipage}

\section{课题介绍}

\skillgroup{博士期间}
{
基于线画的交互式三维几何建模和分析。设计师利用计算机辅助设计系统构建物体\\
概念模型,艺术家通过线条先勾勒出物体的轮廓来设计三维角色,对其而言,建模\\
首先要取得三维线条模型,然后找到三维线框蕴含的三维网格面。线框表示的物体,\\
其蕴含的网格模型非常多,所以具有二义性。[1]中设计了一种自动寻找网格的算法,\\
在目前已有的算法中,是最快最好的。\\
\\
三维网格的获取在工业设计,艺术创作,教育研究等领域有很多应用,比如模拟物体\\
变形,或纹理贴图等。大多数应用,为了实现更好的效果,往往依赖于模型的特征线。\\
位于网格上的封闭的特征线网络(layout),等价于模型的分割,每个分割片的边界位\\
于特征线网络上。作者提出一种提取网格特征网络的算法,其以少数散乱的特征线出\\
发,自动寻找特征线的一个子集,并将期连接成一个封闭的特征线网络,使其代价是\\
所有可能的特征网络中最优的。(未发表内容)\\
\\
鉴于不同应用,需要地特征线网络不同。完全自动的算法,必然无法顾及各不同需求。\\
作者开发了一种交互式的特征线网络提取工具。该工具类似于图像分割工具live-wire,
既用户在模型上定义两个点,算法会自动返回一条光滑的位于几何特征上的线\\
条。[2]中通过定义有别于欧氏度量(isotropic)的各向异性度量(anisotropic),并修改了\\
测地线算法,使其返回在新度量下的测地线,因此保证了线条的光滑性,且位于特征\\
上的。\\
}
\clearpage

\skillgroup{硕士期间}
{研究了分形建模和分形变形的主要技术。分形指具有自身内部相似性的物体,比如\\
树、云、海岸线等。早期的分形模型是严格的自迭代的,既子部分是严格相似于主\\
体部分。更新的研究指向了概率模型,使同一分形可以生成形状各异的分形模型,\\
因此其具有描述更一般物体的能力。分形变形使分形模型“动起来”,但是简单的\\
参数插值会导致变形过程中模型失真,[3,4]提出一种借助多边形凸包来判断分形\\
变形中各个分支可能发生的交叉而引起的失真,从而避免了失真的情形。\\
\\
为了使分形表示更多的物体,比如真实世界的烟花,作者还研究了烟花的分形数学\\
形式,并利用绘理贴图的手法,模拟逼真的烟花。同时利用分形变形思想,实现烟\\
花绽放的过程。
}


%----------------------------------------------------------------------------------------
%	INTERESTS SECTION
%----------------------------------------------------------------------------------------

\section{兴趣爱好}
\begin{tabbing}% Enables tabbing
\hspace{5mm} \= \hspace{14cm} \= \kill % Spacing within the block
\sqbullet \> {篮球, 跑步,羽毛球 } \\% Address line 1
\sqbullet \> {旅行,摄影}\\
\sqbullet \> {历史,军事}
\end{tabbing}%}
%\interestsgroup{
%\interest{跑步,羽毛球, 篮球}
%\interest{旅行,摄影}
%\interest{历史,军事}
%}
%----------------------------------------------------------------------------------------
%	REFEREE SECTION
%----------------------------------------------------------------------------------------

\section{自我评价}

\begin{tabbing}% Enables tabbing
\hspace{5mm} \= \hspace{14cm} \= \kill % Spacing within the block
\sqbullet \> {经过严格的学术训练,有较扎实的科研基础和开阔的视野。善于思考,熟悉算法,有}\\
\> {能力对具体问题进行抽象和数学形式化,并找到较有效且快速的算法解决。编程能力}\\
\> {强,编写过一些较复杂的算法,开发过几个交互式三维几何处理的软件。} \\% Address line 1
\\
\sqbullet \> {参与过几个项目,积累一定的工程经验。}\\
\\
\sqbullet \> {热爱生活,喜欢集体活动。无论是学习中,还是生活中,喜欢和同学、朋友交流,}\\
\> {分享经验,分享喜悦。}
\end{tabbing}%}




%----------------------------------------------------------------------------------------

\end{document} 
